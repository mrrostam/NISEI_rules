\section{Game Concepts}\label{sec:game_concepts}
\subsection{General}
\textit{These rules are compatible with cards from the game \textsc{Android: Netrunner} by Fantasy Flight Games. \textsc{Android: Netrunner} is a game about the cyber struggle between massive Corporations and subversive hackers known as Runners.}
\begin{enumerate}
	\item The game is played between two players. One player takes the role of the Corp~(Corporation) and the other takes the role of the Runner. This rules document will frequently refer to a player interchangeably with their game role.
	\item Each player needs a legal deck, an identity card for their role, and any extra cards used	from outside their deck. They also need a supply of tokens as described in \hyperlink{page.i}{section 1.9}.	The constraints that define the legality of a deck are defined in \hyperlink{page.i}{section 1.4}, and the cases where cards outside the deck and identity can be used are defined in \hyperlink{page.i}{section 1.5}.
\end{enumerate}
\subsection{The Golden Rules}
\begin{enumerate}
	\item If the text of a card directly contradicts these rules, the text of the card takes precedence.
	\item If a rule or ability directs something to happen, but another effect states that it cannot happen, the ``cannot'' ability takes precedence.
	      \begin{enumerate}
		      \item If a ``cannot'' effect prohibits all of the effects of another ability, that ability cannot be triggered.
		      \item If a ``cannot'' effect prohibits only part of another ability, that ability can be triggered,	but the prohibited steps of resolving that ability are not carried out.
	      \end{enumerate}
	      \example{During a run, Lockdown's subroutine fires, preventing the Runner from drawing cards for the remainder of the turn. The Runner has a \card{Diesel} and a \card{Process Automation} in their grip. For the remainder of this turn, they cannot play Diesel as its entire ability is prohibited, but they can play \card{Process Automation}. Even though cards cannot be drawn through \card{Process Automation}, the Runner can play it to gain 2\credit.}
	\item If an instruction includes the words ``if able,'' it can only be carried out fully or not at all. If any part of the instruction is not possible to carry out, the entire instruction is ignored.
	\item If an instruction does not include the words ``if able,'' as much of that instruction as possible is carried out. Any parts of the instruction that are not possible to carry out are ignored.
	\item A player can only take an action or use an ability if its effect has the potential to change the game state. This potential is assessed strictly by what the action or ability will have the ability \textcolor{Bittersweet}{ can be	expected to} accomplish, without regard to the consequences of paying any costs to initiate that action or ability and without regard to any other abilities that may \textcolor{Bittersweet}{ meet their conditions in the process of} initiating or resolving that action or ability.\\[10pt]
	      \example[1]{The Corp has one unrezzed piece of ice installed and \card{Liquidation} in HQ. As \card{Liquidation} requires the Corp to have at least one rezzed card to trash, the Corp cannot play \card{Liquidation} as it cannot change the game state.}\\[10pt]
	      \example[2]{The Corp has one rezzed piece of ice installed and \card{Liquidation} in HQ. Because the Corp has at least one rezzed card that could be trashed (which would change the game state), they can play \card{Liquidation}, spending a click and paying its play cost. The Corp then chooses any number of their rezzed cards to trash, which could be zero cards.}\\[10pt]
	      \example[3]{The Runner is playing \card{Armand ``Geist'' Walker} and has \card{Forger} installed. The Runner can only trash \card{Forger} and trigger Geist's ability when they are about to take a tag (which Forger could avoid) or while they have a tag (which Forger could remove). Using \card{Forger} at any other time has no potential to change the game state.}
\end{enumerate}
\subsection{Symbols}
\begin{enumerate}
	\item Several non-English symbols appear on cards and in this rules document. This section serves as a basic guide to those symbols.
	\item When this document is presented in a format without images, plaintext replacements are used. These replacements are listed along with the symbols themselves for reference.
	\item The symbol \credit (plaintext: \credit) stands for ``credit''. It always appears with a numeral, such as 1\credit, which means ``one credit,'' or 3\credit, which means ``three credits.'' See \hyperlink{page.i}{section 1.10} for rules about credits.
	\item The symbol \click (plaintext: \click) stands for ``click''. Multiple clicks can be represented either by multiple symbols, such as \click\click, or by a numeral and symbol, such as 2\click, both meaning ``two clicks.'' See \hyperlink{page.i}{section 1.11} for rules about clicks.
	\item The symbol \recurring (plaintext: \recurring) stands for recurring credit. It always appears with a numeral, such as 1\recurring, which means ``one recurring credit,'' or 3\recurring, which means ``three recurring credits.'' See \hyperlink{page.i}{section 1.10.5} for rules about recurring credits.
	\item The symbol \link (plaintext: \link) stands for ``link''. It is always used with a quantity, such as 1\link, which means ``1 link.'' See \hyperlink{page.i}{section~10.7} for rules about link.
	\item The symbol \MU (plaintext: \MU) stands for ``memory unit''. It always appears with a quantity, such as 2\MU, which means ``2 memory units.'' See \hyperlink{page.i}{section~1.19} for rules about memory.
	\item The symbol \sub (plaintext: \sub) stands for ``subroutine''. Each symbol marks a single subroutine on a piece of ice. See \hyperlink{page.i}{rule 3.4.7} for information about subroutines.
	\item The symbol \trash (plaintext: \trash) stands for ``trash this card''. It is used as a self-referential cost in card text, such as ``\trash: Draw 2 cards,'' which means ``Trash this
	      card to draw 2 cards.'' See \hyperlink{page.i}{section 1.15} for rules about costs, and \hyperlink{page.i}{section 1.18} for rules about trashing cards.
	\item \textcolor{Bittersweet}{The symbol [interrupt] (plaintext: [interrupt]) stands for ``interrupt timing''. It is used to designate abilities that resolve immediately before other abilities. See section 9.8 for rules about interrupt abilities.}
\end{enumerate}
\subsection{Deck Construction}
\begin{enumerate}
	\item Each player's deck is associated with a single identity card that determines the faction,	minimum deck size, and influence limit of that deck. The identity card may also stipulate other variances from the standard deckbuilding rules.
	\item Each deck must meet all requirements in this section to be legal for play.
	\item The deck must contain at least as many cards as the minimum deck size indicated on the corresponding identity.
	      \begin{enumerate}
		      \item Identity cards, extra cards that begin the game outside the deck, and player aid cards are not part of the deck and are not counted towards the size of the deck.
		      \item There is no maximum deck size.
	      \end{enumerate}
	\item Decks cannot contain identity cards, player aid cards, cards from the wrong side (Corp cards in a Runner deck or vice-versa), or out-of-faction cards that lack influence costs.
	\item Neutral cards and cards that belong to a faction other than the corresponding identity's faction are all considered out-of-faction. The total influence cost of out-of-faction cards in
	      the deck must not exceed the influence limit of that identity.
	      \begin{enumerate}
		      \item The total influence cost of out-of-faction cards is counted by copy and not by name.
	      \end{enumerate}
	      \example{Including a single copy of \card{Diesel} in a non-Shaper deck adds 2 to the total	influence in that deck, while including two copies of \card{Diesel} in a non-Shaper deck adds 4 to the total influence in that deck.}
	\item A Corp deck must contain agendas totalling a certain number of agenda points, as determined by the total number of cards in the deck.
	      \begin{enumerate}
		      \item A deck with 40 to 44 cards must contain 18 or 19 agenda points.
		      \item A deck with 45 to 49 cards must contain 20 or 21 agenda points.
		      \item A deck with 50 to 54 cards must contain 22 or 23 agenda points.
		      \item A deck with more than 54 cards must contain 22 or 23 agenda points, plus an additional 2 agenda points for every full 5 cards in the deck over 50.\\[10pt]
		            \example{A 66 card deck requires 6 additional agenda points, since it includes 3 sets of 5 cards beyond 50. This gives a final requirement of either 28 or 29 agenda points.}
	      \end{enumerate}
	\item The deck cannot contain more than 3 copies of any single card, by name. Some cards stipulate alternative copy limits in their card text.
	\item Tournament play may impose other restrictions or requirements on the legal configurations of identities and decks.
\end{enumerate}
\subsection{Extra Cards}
\begin{enumerate}
	\item Some abilities allow for the use of additional cards from outside the deck.
	\item One Corp identity, \card{Jinteki Biotech: Life Imagined}, has the ability, ``Before taking your first turn, you may \textcolor{Bittersweet}{switch} this card with any copy of Jinteki Biotech.'' There are 3 versions of this identity, which have different abilities on their reverse side.
	      \begin{enumerate}
		      \item A player \textcolor{Bittersweet}{playing} Jinteki Biotech as their identity may bring any number of copies of that identity along with their deck. After completing game setup, that player may choose any copy they own to be their active identity for the duration of the game. All other copies are placed outside the game.
	      \end{enumerate}
	\item One Runner identity, \card{Adam: Compulsive Hacker}, has the ability, ``You start the game with 3 different directive cards installed (these cards are not considered part of your deck).''
	      \begin{enumerate}
		      \item A player \textcolor{Bittersweet}{playing} Adam as their identity must bring at least 3 differently named cards with the directive subtype along with their deck. The player may bring any number of directive cards over the required 3.
		      \item After players reveal their identities, the player \textcolor{Bittersweet}{playing} Adam selects exactly 3 of their provided directive cards. Those cards begin the game installed in the play area. All other directives the player brought this way remain outside the game.
		      \item A player \textcolor{Bittersweet}{playing} Adam can also include directive cards in their deck. The cards chosen to be installed at the beginning of the game do not impact the deck's influence requirements or the maximum allowed number of copies of those cards.
		      \item Once the game has begun, the cards installed this way are treated exactly as any other installed cards. Abilities that move these cards to other zones function normally, including shuffling them into the stack.
	      \end{enumerate}
	\item Two cards, \card{Rebirth} and \card{DJ Fenris}, make use of identity cards other than the one selected by the player during deck construction.
	      \begin{enumerate}
		      \item A player may bring any number of additional Runner identity cards along with their deck. These cards are kept in a pile outside the game. The Runner may look at these cards at any time.
		      \item When an ability refers to an identity other than the Runner's current identity, it refers to the cards provided this way. If an identity card leaves the play area, it must be
		            returned to the pile outside the game.
		      \item In a tournament, all identities brought to the game this way must be legal for players to use as their actual identity in that tournament.
		      \item \textcolor{Bittersweet}{One Runner identity, Hoshiko Shiro, is double-sided (see section 3.1). If the Runner switches their identity with Hoshiko, it enters play with the front side faceup. If the Runner switches Hoshiko with another identity, the new identity enters play faceup regardless of Hoshiko’s current status.}
	      \end{enumerate}
\end{enumerate}
\clearpage
\subsection{Starting the Game}
\begin{enumerate}
	\item The players decide who will play as the Corp and who will play as the Runner. Each player places an appropriate identity card for the side they are to play faceup in the play area, then supplies a corresponding deck, placing it facedown in the play area, and any appropriate extra cards.
	      \begin{enumerate}
		      \item Some identity cards have abilities that affect setup. Even though cards are not active until the game begins, these identities still alter setup at the appropriate step as indicated by their text.
		      \item \textcolor{Bittersweet}{Some identities are double-sided (see section 3.1). If a player is playing a double-sided identity, it begins the game with the front side faceup.}
	      \end{enumerate}
	\item If a player's identity has an ability that affects setup or the start of the game, and that ability does not directly correspond to a setup step outlined in this section of the rules, that player makes any necessary decisions and changes for their identity's special setup or start of game abilities at this time. If both players have setup or start of game abilities at this time, the Corp resolves theirs first.
	\item The players create the bank by gathering all types of tokens described in \hyperlink{page.i}{section 1.9}.
	\item Each player takes five credits from the bank, placing the credits in their credit pool.
	\item Each player shuffles their deck.
	\item Each player draws five cards from the top of their deck to form their starting hand.
	      \begin{enumerate}
		      \item After drawing starting hands, the Corp may choose to a take a mulligan; then, the Runner may choose to take a mulligan. To take a mulligan, the player shuffles their	starting hand back into their deck, then draws a new starting hand. They must keep the second hand as their starting hand.
	      \end{enumerate}
	\item The game starts and the Corp takes their first turn.
	      \begin{enumerate}
		      \item If the Corp's identity has an ability with instructions for ``before taking your first turn,'' the Corp resolves that ability immediately before taking their first turn, and thus before the game starts.
	      \end{enumerate}
\end{enumerate}
\subsection{Ending the Game}
\begin{enumerate}
	\item The game ends when a player meets one of their win conditions.
	      \begin{enumerate}
		      \item If both players would simultaneously satisfy their win conditions, the game ends in a draw.
	      \end{enumerate}
	\item Each player has two possible win conditions available.
	      \begin{enumerate}
		      \item Either player can win by collecting agenda points in their score area (usually done by scoring or stealing agendas; see \hyperlink{page.i}{section 1.16}). A player with a score of 7 or more wins the game.
		      \item The Corp wins if the Runner is flatlined. The Runner is flatlined immediately if they suffer more damage than they have cards in their grip. The Runner is also flatlined if, at the beginning of their discard phase, their maximum hand size is less than 0. See \hyperlink{page.i}{section 10.4} for more information about damage.
		      \item The Runner wins if the Corp is required to draw a card from R\&D but cannot because R\&D is empty.
	      \end{enumerate}
\end{enumerate}
\subsection{Cards}
\begin{enumerate}
	\item There are six types of Corp cards: agendas, assets, ice, identities, operations, and upgrades. All cards except the identity card are shuffled into the Corp's deck at the beginning of the game.
	\item There are five types of Runner cards: events, hardware, identities, programs, and resources. All cards except the identity card are shuffled into the Runner's deck at the beginning of the game.
	\item Cards that are ACTIVE are able to affect the game through their abilities.
	      \begin{enumerate}
		      \item Runner cards that are installed and faceup in the play area, Corp cards that are installed and rezzed, events and operations in the play area, agendas in the Corp's score area, and both players' identities are active.
		      \item Unless otherwise stated, Runner cards are played and installed active into the play area.
		      \item Unless otherwise stated, Corp cards other than operations are installed unrezzed,	and thus inactive, into the play area. Assets, ice, and upgrades made active by rezzing them; agendas are made active by scoring them. Operations are played active into the play area.
	      \end{enumerate}
	\item Cards that are INACTIVE are unable to affect the game or have most of their abilities	used.
	      \begin{enumerate}
		      \item Cards are inactive in R\&D, HQ, Archives, the heap, the grip, the stack, and the Runner's score area, \textcolor{Bittersweet}{while set aside,} and while removed from the game. Cards installed facedown are also inactive.
		      \item Inactive cards in most zones retain their printed characteristics (name, card type, faction, cost, subtypes, influence, etc). Runner cards installed facedown have no characteristics.
	      \end{enumerate}
	\item \textcolor{Bittersweet}{For a player to DRAW 1 or more cards is to take that many cards from the top of their deck and put them into their hand. See section 4.3 and section 4.4 for information about decks and hands.}
	\begin{enumerate}
		\item If a replacement effect modifies the act of adding a card to a player’s hand,	and that replacement effect is applied to a card being drawn, the card still counts as drawn.

	\end{enumerate}
	\item Most abilities are active if and only if the card they appear on is active. \hyperlink{page.i}{Rule 9.1.6} details the cases where abilities on inactive cards are still active.
	\item Abilities can convert cards from one type into another type, or even from a card into a counter. \hyperlink{page.i}{Rules 10.1.3} and \hyperlink{page.i}{10.1.4} explain the details of card conversion.
\end{enumerate}
\subsection{Counters and Tokens}
\begin{enumerate}
	\item COUNTERS and TOKENS are game pieces (or equivalent) that track various resources, effects, and statuses of players and their cards.
	      \begin{enumerate}
		      \item The terms ``counter'' and ``token'' are interchangeable.
	      \end{enumerate}
	\item The BANK is the supply of tokens not yet in play. Tokens in the bank are available to both players to take and use as dictated by the game rules and card abilities. Players do not control tokens in the bank and cannot spend them. The bank is an unlimited supply; running out of game pieces to track a type of token does not prohibit a player from \textcolor{Bittersweet}{gaining tokens of that type or placing them on cards}.
	\item Some abilities can convert a card into a counter. A counter put into play in this way is tracked \textcolor{Bittersweet}{with} the card \textcolor{Bittersweet}{acting} as a game piece. See \hyperlink{page.i}{rule 10.1.4}.
	\item Some abilities can cause a player or card to have or be ``considered to have'' one or more tokens or additional tokens without placing or giving those tokens. A token considered to exist in this way conveys all the same information and effects as a token of the specified type would, except that it cannot be moved or removed from the card or player it belongs to, nor can it be spent to pay a cost.
	\item There are ten types of tokens: credit tokens, click tokens, tag tokens, bad publicity tokens, brain damage tokens, advancement tokens, virus counters, power counters, agenda counters, and condition counters.
	      \begin{enumerate}
		      \item CREDIT TOKENS are used to track the number of credits each player has in their credit pool; they can also be placed on cards. Rules for credits, the credit pool, and	other related concepts are in \hyperlink{page.i}{section 1.10}.
		      \item  CLICK TOKENS are a gameplay aid used to track the clicks the active player has spent or has left to spend during their turn. Rules for clicks are in \hyperlink{page.i}{section~1.11}.
		      \item  TAG TOKENS are used to represent tags on the Runner. Rules for tags are given in \hyperlink{page.i}{section~10.5}.
		      \item  BAD PUBLICITY TOKENS represent bad publicity the Corp has earned. Rules for bad publicity are in \hyperlink{page.i}{section 10.6}.
		      \item BRAIN DAMAGE TOKENS are a gameplay aid used to track brain damage the Runner has suffered. Each point of brain damage the Runner suffers forces them to take a brain damage token. See \hyperlink{page.i}{section 10.4}.
		      \item ADVANCEMENT TOKENS are a token used primarily on installed agendas to track the Corp's progress toward being able to score them.
		      \item VIRUS COUNTERS are a generic token, used primarily by virus programs, that can be removed by the Corp purging them. See \hyperlink{page.i}{rule 10.1.1}.
		      \item POWER COUNTERS are a generic token used by a variety of cards. They have no special rules.
		      \item AGENDA COUNTERS are a generic token used primarily on scored agendas. They  have no special rules.
		      \item CONDITION COUNTERS are tokens that have rules text. Their abilities are active as	long as they are hosted on a card.
	      \end{enumerate}
\end{enumerate}
\subsection{Credits}
Each of the Runner's credits represents enough money to upgrade some basic parts for their console, have a meal at a decent restaurant, or buy a ticket and some concessions for a night at the sensies.
Each of the Corp's credits represents enough money to manufacture a run of computer parts, buy out a decent restaurant, or film a low-budget sensie.
\begin{enumerate}
	\item A CREDIT (\credit) is the basic unit of currency. Players spend their credits to pay for various costs, card abilities, traces, etc. Credit tokens most commonly represent 1\credit each, but can represent larger denominations if clearly marked.
	\item  Each player has a CREDIT POOL where they keep a supply of credit tokens matching the credits they have available to spend. The number of credits in a player's credit pool is open information.
	\item Credits enter and leave a player's credit pool as that player gains, spends, or loses credits.
	      \begin{enumerate}
		      \item A player GAINS credits whenever credits enter their credit pool from any location.
		      \item If a player is instructed to LOSE credits, that player is forced to move the specified number of credits from their credit pool to the bank. Players cannot lose credits from cards.\\[10pt]
		            \example{If a subroutine on \card{DNA Tracker} resolves, the Runner must lose 2\credit~from their credit pool, or 1\credit~if that is all they have. If the Runner's credit pool is empty, the effect does nothing.}
		      \item If a player is instructed to SPEND or PAY credits, that player must put the specified	number of credits back into the bank. The credits spent can come from the credit pool or from a card that player controls with an ability allowing credits on it to be spent for the type of ability being resolved. \textcolor{Bittersweet}{The terms ``spend'' and ``pay'' are synonymous.}\\[10pt]
		            \example[1]{The Runner may use the credit from \card{Cyberfeeder} to pay for Atman's first ability.}\\[10pt]
		            \example[2]{The Runner may use credits from Ghost Runner when secretly spending credits to resolve a Psi ability (such as The Future Perfect).}\\[10pt]
		            \example[3]{The Runner must use credits from \card{Ghost Runner} if this is the only way for them to pay 3\credit when resolving the ``when encountered'' ability on \card{Tollbooth}.}
	      \end{enumerate}
	\item Abilities can place credits on cards \textcolor{Bittersweet}{as discussed in section 1.12}. Credits on a player's card are not in that player's credit pool; the player can only interact with those credits as instructed by card abilities.
	      \begin{enumerate}
		      \item If a player is instructed to TAKE credits from a card, they remove that many credits from that card and gain those credits.
		      \item If an ability specifies how a player is allowed to spend credits from a card, they can be spent from that card as if they were in the player's credit pool for the specified actions or abilities.
		      \item If an ability specifies a time period during which a player is allowed to spend credits from a card, they can be spent from that card as if they were in the player's credit pool for any purpose (other than losing credits) as long as the specified time period is active or the specified duration has not expired.
		      \item Spending credits from a card is considered ``using'' the card \textcolor{Bittersweet}{that allowed those credits to be spent. See section 9.1.6}.
	      \end{enumerate}
	\item \additemtotoc{Recurring Credits}
	      \begin{enumerate}
		      \item RECURRING CREDITS (\recurring) place credits on a card repeatedly. The text ``N\recurring'' means ``Before this card becomes active or your turn begins, if there are fewer than N credits on it, place credits on this card until there are N credits on it.''
		      \item Recurring credits do not accumulate. They are refilled only up to the indicated number.
		      \item Recurring credits are refilled during \hyperlink{page.i}{step 5.6.1c} of the Corp's turn and \hyperlink{page.i}{step 5.7.1c} of the Runner's turn, before other abilities that \textcolor{Bittersweet}{apply} at the start of the turn resolve.\\[10pt]
		            \example{The Runner installs Spinal Modem, which has 2\recurring. 2 credits are placed on Spinal Modem now. The Runner spends 1 of those credits later in their turn. At the beginning of the Runner's next turn, the credits on Spinal Modem should be replenished up to 2\credit, so one more credit is placed on it.}
	      \end{enumerate}
\end{enumerate}
\subsection{Clicks}
Working a job, making connections, and especially jacking in--everything you do takes time, and it always goes by faster than you think. A click represents an abstract amount of time spent on a
particular activity, either several hours all at once or scattered across the day.
\begin{enumerate}
	\item A CLICK (\click) is the basic unit of activity. Players spend their clicks to perform actions and trigger abilities. Each click token represents 1 click.
	\item As the first step of a player's turn, they gain an allotted number of clicks to spend during the action phase of that turn. See section 5 for details about the procedures of player turns.
	      \begin{enumerate}
		      \mrule{The Corp receives 3\click on each of their turns.}
		      \item The Runner receives 4\click on each of their turns.
	      \end{enumerate}
	\item Some cards have the subtype priority and the text, ``Play only as your first \click.'' A player can only play a priority card using the basic action to play an event or operation, and only if they have not spent any other clicks that turn. Losing clicks does not affect a player's ability to play a priority card.
\end{enumerate}